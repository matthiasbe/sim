\documentclass[9.5pt]{beamer}

\mode<presentation>
{
  \usetheme{Warsaw}       % or try default, Darmstadt, Warsaw, ...
  \usecolortheme{default} % or try albatross, beaver, crane, ...
  \usefonttheme{serif}    % or try default, structurebold, ...
  \setbeamertemplate{navigation symbols}{}
  \setbeamertemplate{caption}[numbered]
} 

\usepackage[utf8]{inputenc} % accents 8 bits dans le fichier
\usepackage[T1]{fontenc}      % accents codés dans la fonte
\usepackage[french]{babel}
\usepackage{amsmath,amssymb}
\usepackage{graphicx}
\usepackage{fancyhdr}
\usepackage{siunitx}
\usepackage[mode=buildnew]{standalone}


% Here's where the presentation starts, with the info for the title slide
\title[GLCS : Projet simulation de chaleur]{Présentation GLCS \\Projet simulation de chaleur}
\author[\bsc{Beaupère} \& \bsc{Granger}]{Matthias \bsc{Beaupère} \& Pierre \bsc{Granger}}
\institute{M2 CHPS}
\date{\today}

\begin{document}
\setbeamercolor{captioncolor}{fg=white,bg=red!80!white}
\setbeamertemplate{caption}{%
\begin{beamercolorbox}[wd=0.8\linewidth, sep=.2ex]{captioncolor}\tiny\centering\insertcaption%
\end{beamercolorbox}%
}

\begin{frame}
  \titlepage
\end{frame}

\begin{frame}{Plan}
	\tableofcontents[hideallsubsections]
\end{frame}

\section{Intro}
	\begin{frame}{Frame}

	\end{frame}

\section{Présentation de l'algorithme}
	\begin{frame}{La méthode des itérations simultanées}

	\end{frame}

\section{Séquentiel}
	\subsection{Description}
		\begin{frame}{Description de l'algorithme}

		\end{frame}

	\subsection{Performances théoriques}
		\begin{frame}{Performances théoriques}

		\end{frame}

	\subsection{Performances pratiques}
		\begin{frame}{Performances pratiques}

		\end{frame}

	\subsection{Locking}
		\begin{frame}{Locking}

		\end{frame}

\section{Multic\oe{}urs}
		\subsection{Description}
		\begin{frame}{Multic\oe{}urs}

		\end{frame}

	\subsection{Performances théoriques}
		\begin{frame}{Multic\oe{}urs : performances théoriques}

		\end{frame}

	\subsection{Performances pratiques}
		\begin{frame}{Multic\oe{}urs : performances pratiques}

		\end{frame}

\section{Multin\oe{}uds}
	\subsection{Description}
		\begin{frame}{Multin\oe{}uds}

		\end{frame}

	\subsection{Performances théoriques}
		\begin{frame}{Performances théoriques}

		\end{frame}

	\subsection{Performances pratiques}

		\begin{frame}{Multin\oe{}uds : performances pratiques}

		\end{frame}

\section{Conclusion}
	\begin{frame}{Conclusion}

	\end{frame}

\end{document}
